%%
%% Copyright (c) 2018-2024 Weitian LI
%% CC BY 4.0 License
%%

% Chinese version
\documentclass[zh]{resume}

% File information shown at the footer of the last page
\fileinfo{%
  % \faCopyright{} 2023--2024, Xxx Xx \hspace{0.5em}
  %\creativecommons{by}{4.0} \hspace{0.5em}
  %\githublink{account}{repo} \hspace{0.5em}
  % \faEdit{} \today
}

\name{宏伟}{赵}

% \keywords{关键词1, 关键词2, ...}

% \tagline{<current-position>}
% \tagline{\texorpdfstring{\icon{\faBinoculars} }{}<position-to-look-for>}
% \tagline{\texorpdfstring{\icon{\faBinoculars} }{}期望岗位}

% \photo[
%   shape=<circular|square>,    % default is circular
%   position=<left|right>,      % default is left
% ]{<width>}{<filename>}
% Example:
\photo[shape=square]{7em}{myself.jpg}

\profile{
  \mobile{13263382922}
  \email{izhw1024@163.com}
  \github{Geeks-Z} \\
  \university{北京航空航天大学}
  \degree{计算机软件与理论}
  % \birthday{yyyy-mm-dd}
  \address{政治面貌:中共党员}
  % Custom information:
  % \icontext{<icon>}{<中共党员>}
  % \iconlink{<icon>}{<link>}{<text>}
}

\begin{document}
\makeheader

%======================================================================
% Summary & Objectives
%======================================================================
% \begin{abstract}
% 物理学专业(射电天文方向)直博研究生,有扎实的物理、数学与统计学基础,
% 擅长数据建模与分析,热衷计算机和网络技术,
% 有 10 年的 Linux 和 BSD 使用经验,熟练掌握 Shell、Python 和 C 语言编程。
% 积极实践自由开源精神,
% 在 \link{https://github.com/liweitianux}{GitHub} 上分享多个项目,
% 是 \link{https://www.dragonflybsd.org}{DragonFly BSD} 操作系统的开发者,
% 并积极参与其他多个开源项目。
% \end{abstract}

%======================================================================
\sectionTitle{教育背景}{\faGraduationCap}
%======================================================================
\begin{educations}

    \education%
    {至今}%
    [2021.09]%
    {北京航空航天大学}%
    {计算机学院}%
    {计算机软件与理论}%
    {博士}
  \separator{0.5ex}
  \education%
    {2021.03}%
    [2018.09]%
    {南京航空航天大学}%
    {民航学院}%
    {交通运输规划与管理}%
    {硕士}

  \separator{0.5ex}
  \education%
    {2013.09}%
    [2017.06]%
    {吕梁学院}%
    {矿业工程系}
    {安全工程}%
    {本科}
\end{educations}

% %======================================================================
\sectionTitle{研究方向}{\faBriefcase}
% %======================================================================
\begin{experiences}
  \experience%
    [至今]%
    {2021.09}%
    {\bf \color{red} 面向开放环境的增量式图像识别算法研究}%
    [\begin{itemize}
    \item 针对{\bf 开放环境}中动态多样的{\bf 数据分布和模态}难题,基于{\bf 大规模预训练模型},设计图像识别模型。核心在于研究{\bf 持续学习}机制,使模型能增量式学习新数据的同时,有效保留对旧知识的稳定记忆(即解决{\bf 灾难性遗忘问题}),从而显著提高模型在开放世界中的泛化能力和适应性。
    % \begin{itemize}
    \item 基于MoE和LoRA混合微调PTM,解决Prompt超参和Adapter计算量问题。
    \item 设计基于LoRA的多教师蒸馏策略,解决小样本增量学习中的灾难性遗忘和过拟合。
    \item 基于LLM设计层级Prompt生成器,解决多模态增量学习中灾难性遗忘和零样本能力。
  % \end{itemize}
    \end{itemize}]
\separator{0.5ex}
  \experience%
    [2021.03]%
    {2018.09}%
    {\bf \color{red} 通用航空旋翼无人机路径规划研究}%
    [\begin{itemize}
      \item {\bf 精确建模降复杂度}:采用栅格法将目标区域三维点云映射到矩阵,实现障碍物精确建模。
      \item {\bf 能耗最低为目标}:建立能耗最低目标函数,纳入无人机性能与自然风约束的航迹规划模型。
      \item {\bf 混合算法提效率}:设计了改进蚁群和A*混合算法,航迹规划时间相比传统A*降低了96.6\%。
    \end{itemize}]
\end{experiences}

%======================================================================
\sectionTitle{科研成果}{\faAtom}
%======================================================================
\begin{itemize}
  \item  
  {\bf Hongwei Zhao}, Rui Liu, Yansong Liu, Zhiyuan Zou, Yong Chen; Dynamic LoRA-Experts and Prototype-Ensemble Matching for Class-Incremental Learning. International Journal of Computer Vision(IJCV). ({\bf \color{red}CCF A, Q1, 计算机视觉领域Top期刊, IF影响因子: 9.3, 一作,} Under Review)

  \item  
  {\bf Hongwei Zhao}, Rui Liu, Ruoheng Li, Yong Chen; Decoupled and Distilled: Task-Adaptive LoRA-Teachers with Ensemble Knowledge Transfer for Few-Shot Class-Incremental Learning. Pattern Recognition(PR). ({\bf \color{red}CCF B, Q1,计算机视觉领域Top期刊, IF影响因子: 7.6, 一作,} Under Review)

  \item  
  {\bf Hongwei Zhao}, Rui Liu, Yong Chen; Decoupling Stability and Plasticity: A Dynamic Retrieval Framework for Rehearsal-Free Class-Incremental Learning. IEEE International Conference on Multimedia and Expo(ICME). ({\bf \color{red}CCF B, 一作,} Under Review)

  \item  
  {\bf Hongwei Zhao}, Rui Liu, Yong Chen; Description-Aligned Layer-wise Instance Prompts for Exemplar-Free Vision-Language Continual Learning. IEEE International Conference on Multimedia and Expo(ICME). ({\bf \color{red}CCF B, 一作,} Under Review)

  \item Zheng Huang, Xuefeng Zhai, Hongxing Wang, Hang Zhou, {\bf Hongwei Zhao}, Mingduan Feng; On the 3D Track Planning for Electric Power Inspection Based on the Improved Ant Colony Optimization and A * Algorithm. Mathematical Problems in Engineering, 2020, 2020(1): 8295362. ({\bf \color{red}SCI, Q1, 通讯作者 })
  
  \item 黄郑, 王红星, 周航, 张星炜, {\bf 赵宏伟}; 基于混合算法的电力杆塔巡检实时航迹规划. 中国电力, 2021, 54 (11): 214-220.({\bf \color{red}北大中文核心,学生一作})
\item 黄郑, 王红星, 潘志新, 周航, 宋煜, 黄祥, 张星炜, {\bf 赵宏伟}; 一种多旋翼无人机巡检路径规划方法,202010374783.5;({\bf \color{red}已授权,国家专利,学生一作})
    
\end{itemize}

%======================================================================
\sectionTitle{项目经历}{\faCode}
%======================================================================
% {\bf 科技部重大专项}
\begin{itemize}
  \item {\bf \color{red} \link{https://nrii.org.cn/}{\texttt{重大科研基础设施和大型科研仪器国家网络管理平台(科技部重大专项)}}\quad  学生第一负责人} \\
  {\bf 项目简介}:近年来,我国科研设施与仪器规模持续增长,但利用率低、重复建设、闲置浪费等问题突出。为提升资源利用效率和服务潜能,国务院发布《关于国家重大科研基础设施和大型科研仪器向社会开放的意见》,要求建立开放共享制度与统一网络管理平台。目前,国家网络管理平台纳入了600多类、10万多台套(价值近{\bf 1600亿})设施和仪器并向社会开放,同时,整合了31个省市、25个部门,近{\bf 5000家单位}的平台系统,年访问量近{\bf 800万次}。\\
  {\bf 主要工作}:(1) 开展系统对接与数据核验工作;(2) 完成平台前后端改版升级;(3) 协助老师统筹技术方案和人员安排,对接多家单位沟通需求,完成商务/技术文件撰写及合同签订。
  \item {\bf \color{red} \link{https://sg.nrii.org.cn/}{\texttt{大型科研仪器购置系统(科技部重大专项)}} \quad {\bf 学生第一负责人}} \\
  {\bf 项目简介}:大型科研仪器购置系统是国家网络管理平台上申购单位进行仪器购置申请的统一途径,是查重评议系统的数据来源,也是科技部、财政部等主管部门进行仪器申购管理的有力工具。大型科研仪器购置系统按照国家网络管理平台的统一数据标准,规范了各单位的申报流程,支持申购单位仪器购置申请和申诉,为管理部门提供任务管理和结果公式等功能。\\
  {\bf 主要工作}:(1) 制定系统设计方案,采用Vue+SpringBoot前后端分离技术;(2) 统筹人员安排,对接甲方需求,控制项目进度;(3) 完成商务/技术文件撰写及合同签订。
  % \item {\bf \link{https://irshare.cn/v2/}{\texttt{深圳市重大科技基础设施和大型科研仪器共享平台}} \quad {\bf 学生第一负责人}} \\
  % {\bf 项目简介}:大型科研仪器购置系统是国家网络管理平台上申购单位进行仪器购置申请的统一途径,是查重评议系统的数据来源,也是科技部、财政部等主管部门进行仪器申购管理的有力工具。大型科研仪器购置系统按照国家网络管理平台的统一数据标准,规范了各单位的申报流程,支持申购单位仪器购置申请和申诉,为管理部门提供任务管理和结果公式等功能。\\
  % {\bf 主要工作}:(1) 制定系统设计方案,采用Vue+SpringBoot前后端分离技术;(2) 统筹人员安排,对接甲方需求,控制项目进度;(3) 完成商务/技术文件撰写及合同签订。
  \item {\bf \color{red} 无人机智能巡检关键技术与三维平台应用\quad  学生第一负责人} \\
{\bf 项目简介}:本项目运用机载激光雷达完成对江苏省电力线路及杆塔的高精度全息数字建模。基于三维点云数据实现多旋翼无人机对电力杆塔巡检实时的厘米级航迹规划。最后,整合研发出一套完整的电力杆塔巡检系统。实现电力杆塔的数字化、自动化、精细化巡检。\\
{\bf 主要工作}:(1)基于三维点云数据,结合无人机悬停精度,通过ArcGIS接口实现飞行空间栅格化建模;(2)构建最低能耗与最小光照代价的目标函数,考虑自然风与光照影响;(3)改进蚁群和A*混合算法,解决不同时间/风向下的实时路径规划;
% (4)开发电力巡检系统原型工具,包括路径规划、安全分析、知识库、参数库及可视化引擎;(5)开发电力巡检系统原型工具,包括路径规划、安全分析、知识库、参数库及可视化引擎。

\item {\bf \color{red} 代码调试和运行时监控技术(装备发展部信息统计局“十三五”预研课题)\quad  学生第一负责人} \\
{\bf 项目简介}:
% 本项目主要面向天熠集成开发环境,研究了基于天熠嵌入式操作系统的多任务并行程序的代码调试和运行时监控技术。在Eclipse插件开发平台的基础上,基于程序插桩、上下位机网络通信、SWT等技术,围绕天熠嵌入式操作系统下的并行程序过程重现、动态跟踪及精确定位、实时监控和分析等关键模块,最终实现了过程重现的程序调试、精确的错误现场追踪定位,以及一种低开销的程序运行时状态实时监控分析代码调试原型工具,为构建高质量的并行软件提供了技术保障和应用支持。
本项目研究了基于天熠系统的多任务并行程序的调试和运行时监控技术。在Eclipse基于程序插桩、上下位机网络通信、SWT等技术,围绕天熠系统下的并行程序过程重现、动态跟踪及精确定位、实时监控和分析等关键模块,实现了过程重现的程序调试、精确的错误现场追踪定位,低开销的程序运行时状态实时监控分析代码调试原型工具,为构建高质量的并行软件提供了技术保障和应用支持。
\\
{\bf 主要工作}:(1)设计与实现并行程序调试工具的界面与操作;(2)
% 设计与实现并行程序错误检测分析、龙芯3A目标机运行状态实时监控、主机与目标机网络通信等模块;(3)
编制软件设计说明等相关文档。
\end{itemize}

% %======================================================================
\sectionTitle{荣誉奖项}{\faCogs}
% %======================================================================
\begin{itemize}
  \item 国家奖学金({\bf \color{red}国家级},2016)
  \item 全国大学生数学竞赛三等奖({\bf \color{red}国家级},2016)
  \item 国家励志奖学金(2015)
  \item 吕梁学院院长奖(校级,{\bf \color{red}全校仅20人},2016)
  \item 吕梁学院优秀毕业生(校级,2016)
\end{itemize}


%======================================================================
\sectionTitle{技能与特长}{\faWrench}
% %======================================================================
\begin{competences}
  \comptence{操作系统}{%
    \icon{\faLinux} Linux,
  }
  \comptence{编程\&工具}{%
    Python, PyTorch, C, Java, C++, JavaScript, jQuery, Vue, SSH, Git
  }
  % \comptence{工具}{%
   
  % }
  \comptence{数据分析}{%
    Pandas, Matplotlib, Scikit-learn
  }
  \comptence{\icon{\faLanguage} 语言}{ CET6
  }
  \comptence{特长-羽毛球}{ 北京航空航天大学羽毛球{\bf 校级普通生队队员}, {\bf \color{red} 蝉联}北京航空航天大学羽毛球团体赛{\bf \color{red} 年度总冠军}(2023、2024),2025年“航羽杯”春季羽毛球联赛亚军。
  }
\end{competences}

\end{document}
